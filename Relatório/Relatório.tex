\documentclass{llncs}

\RequirePackage[T1]{fontenc}
\RequirePackage[utf8]{inputenc}

\begin{document}

\title{Trabalho Prático Nº2 – Desenho e implementação de um jogo distribuído na Internet}
\author{Jorge Ferreira \and José Pereira \and  Pedro Cunha}
\institute{Universidade do Minho, Departamento de Informática, 4710-057 Braga, Portugal
\email{\{a64293,a67677,a67680\}}@alunos.uminho.pt}
\maketitle

\begin{abstract}
Neste relatório iremos apresentar a nossa resolução do trabalho prático de Comunicação por Computadores.
Vamos descrever o essencial do projecto assim como a sua implementação e algumas decisões tomadas. No fim vamos apresentar alguns testes da aplicação.
\end{abstract}

\section{Introdução}
Este trabalho requer que seja desenvolvido um jogo baseando-se na distribuição de conteúdos com interação dos utilizadores. A aplicação não pode basear-se em tecnologia baseada em http logo deverá implementar comunicações entre servidores e comunicações UDP entre clientes e servidores. O objetivo do jogo é vários jogadores jogarem um desafio de perguntas de escolha múltipla onde o vencedor é o utilizador que acertar mais perguntas sendo que o critério de desempate é o tempo de resposta às questões certas. No fim do jogo a pontuação é integrada no score global do utilizador.

\section{Implementação}
O ponto fundamental neste trabalho é a comunicação entre cliente e servidor. Para a mesma foram criados PDU's para regular a comunicação entre os intervenientes. Os PDU's implementados são uma versão alterada dos apresentados no enunciado, de forma a incluir por exemplo o nome do utilizador ao criar um desafio.
\subsection{Bibliotecas Utilizadas}
As bibliotecas/classes externas que foram utilizadas para a realização do projecto foram as seguintes:
\begin{itemize}
\item \emph{Apache.IOUtils} - Foi utilizada para utilizar o método que transforma um ficheiro num array de Bytes. As restantes classes da biblioteca Apache foram incluídas para evitar possíveis situações de incompatibilidade.
\item \emph{JLGui.BasicPlayer} - Foi utilizada para reproduzir a Música, todas as restantes bibliotecas dos packages jlgui foram incluídos para evitar possíveis situaçoes de incompatibilidade.
\item \emph{Business.Picture} - Foi utilizada para apresentar Imagem sobre uma janela de Java Swing.
\item \emph{Business.Crono} - Foi utilizada para contar os segundos entre o start() e stop().
\item \emph{Business.Input} - Foi utilizada para possuír métodos robustos e protegidos de Excepções para Input do teclado.
\item \emph{Business.Menu} - Foi utilizada para apresentar menus em formato texto.
\end{itemize}
\section{Modo de Funcionamento}
\subsection{Servidor}
Ao iniciar o servidor, este irá carregar os dados de imagens, texto e perguntas existentes (dos ficheiros Imagens.txt, Musicas.txt e Questoes.txt) para uma estrutura interna que servirá de base para todos os Clientes. De seguida entra em ciclo infinito à espera de mensagens vindas de clientes.\\
Ao receber um pacote de um cliente, o servidor irá primeiro verificar o seu cabeçalho para saber qual é o tipo de ação que deve realizar e, de seguida irá processar o pedido e enviar a resposta correspondente.\\
Para gerar um desafio é necessário que existam mais de 10 perguntas na estrutura e os desafios serão gerados com 10 perguntas aleatórias da base de dados.
\subsection{Cliente}
Um cliente ao iniciar irá enviar logo um pacote `Hello`, ficando à espera que o servidor responda com OK. Caso isso não aconteça, um novo pacote Hello é enviado até ser recebido um Ok.
De seguida irá ser apresentado o menu de início de sessão/registo. Caso o utilizador pretenda registar-se deverá colocar o seu nome, alcunha e palavra passe. Será enviado um pacote `Register` com os três campos referidos, ficando à espera de um OK por parte do servidor. Caso haja erro o utilizador deverá inserir uma alcunha diferente. Caso o utilizador pretenda realizar o login no jogo deverá inserir a alcunha e a palavra passe. Será enviado um pacote `Login` com os campos referidos, ficando à espera que o servidor envie uma resposta com o nome e a pontuação actual. Neste momento encontra-se o menu inicial:
\begin{itemize}
\item "Criar Desafio" - Irá pedir ao utilizador que escolha o nome para o desafio, envia um MakeChallenge com o nome de utilizador e nome de desafio, espera um OK por parte do servidor e executa o menu de Desafio.
\item "Ver Classificações" - Envia um List Rankings, espera pela resposta por parte do servidor e apresenta a resposta.
\item "Listar Desafios" - Envia um ListChallenges, espera pela resposta do servidor e apresenta a mesma.
\item "Realizar Desafio" - Irá pedir ao utilizador que escolha o nome do desafio que pretende realizar, será enviado um AcceptChallenge, esperando um OK do servidor e executa o Menu de Desafio.
\item ''Sair'' - Irá enviar um pacote Logout ao servidor e espera um OK.
\end{itemize}
Caso se encontre no menu de Desafio, as seguintes opções são possíveis:
\begin{itemize}
\item "Modo Jogo" - Entrará no modo de jogo, em que irão ser carregadas e apresentadas, uma a uma as perguntas que perfazem o desafio (Música e Imagem incluidas). O utilizador terá sempre a opçao de desistir do desafio, premindo em 0, e será enviado um pacote Quit e a pontuação obtida até ao momento será descartada. No caso de se tratar de uma resposta a uma pergunta irá ser enviado um pacote com a alcunha do utilizador, o índice da pergunta no desafio, o nome do desafio e a opção escolhida.
\item "Terminar Desafio" - Para terminar o desafio é necessário que o utilizador que o termina seja o utilizador que o criou. Será enviado um pacote EndChallenge com o nome do Utilizador actual e o nome do Desafio a terminar, esperando um OK.
\item "Eliminar Desafio" - Será enviado um pacote Delete com o nome do Desafio, para o poder fazer é necessário que o desafio a eliminar se encontre nos desafios já terminados.
\end{itemize}

\section{Conclusão}


\end{document}