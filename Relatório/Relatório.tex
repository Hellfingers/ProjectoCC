\documentclass{llncs}

\RequirePackage[T1]{fontenc}
\RequirePackage[utf8]{inputenc}

\begin{document}

\title{Trabalho Prático Nº2 – Desenho e implementação de um jogo distribuído na Internet}
\author{Jorge Ferreira \and José Pereira \and  Pedro Cunha}
\institute{Universidade do Minho, Departamento de Informática, 4710-057 Braga, Portugal
\email{\{a64293,a67677,a67680\}}@alunos.uminho.pt}
\maketitle

\begin{abstract}
Neste relatório iremos apresentar a nossa resolução do trabalho prático de Comunicação por Computadores.
Vamos descrever o essencial do projecto assim como a sua implementação e algumas decisões tomadas. No fim vamos apresentar alguns testes da aplicação.
\end{abstract}

\section{Introdução}
Este trabalho requer que seja desenvolvido um jogo baseando-se na distribuição de conteúdos com interação dos utilizadores. A aplicação não pode basear-se em tecnologia baseada em http logo deverá implementar comunicações entre servidores e comunicações UDP entre clientes e servidores. O objetivo do jogo é vários jogadores jogarem um desafio de perguntas de escolha múltipla onde o vencedor é o utilizador que acertar mais perguntas sendo que o critério de desempate é o tempo de resposta às questões certas. No fim do jogo a pontuação é integrada no score global do utilizador.

\section{Implementação}
\subsection{Bibliotecas Utilizadas}
As bibliotecas externas que foram utilizadas para a realização do projecto foram as seguintes:
\begin{itemize}
\item \emph{Apache.IOUtils} - Foi utilizada para utilizar o método que transforma um ficheiro num array de Bytes. As restantes classes da biblioteca Apache foram incluídas para evitar possíveis situações de incompatibilidade.
\item \emph{JLGui.BasicPlayer} - Foi utilizada para reproduzir a Música.
\end{itemize}
\section{Modo de Funcionamento}
\section{Conclusão}


\end{document}